\chapter{Making our first product delivery}

\section*{Flows in the system}

Object oriented programming sucks.
That is why it was decided to build the product
with the more functional component composition paradigm.
Consequently, these UML diagrams are a little more abstract
than they would be diagramming for object-oriented software.

\begin{figure}[h]
	\centering
	\begin{sequencediagram}
		\newthread{volunteer}{Volunteer}
		\newinst[1.8]{ui}{:App}
		\newinst[1.2]{shifts}{:Shifts}
		\newinst[1.2]{store}{:Store}
		
		\begin{call}{volunteer}{Go to shifts}{ui}{Available shifts}
			\begin{call}{ui}{showShifts()}{shifts}{listShifts()}
			\begin{call}{shifts}{getShifts()}{store}{shifts}
			\end{call}
			\end{call}
		\end{call}

		\begin{call}{volunteer}{Select shift}{ui}{Confirmation}
			\begin{call}{ui}{takeShift()}{store}{success}
			\end{call}
		\end{call}
	\end{sequencediagram}
	\caption{Take shift}
\end{figure}


\begin{figure}
	\centering
	\begin{sequencediagram}
		\newthread{volunteer}{Volunteer}
		\newinst[1.5]{app}{:App}
		\newinst[2]{portal}{:Portal}
		\newinst[1]{store}{:Store}
		
		\begin{call}{volunteer}{First time visit}{app}{Login page}
			\begin{call}{app}{showRegistration()}{portal}{Login page}
			\begin{call}{portal}{register()}{store}{success}
			\end{call}
			\end{call}
		\end{call}
	\end{sequencediagram}
	\caption{Create user}
\end{figure}


\begin{figure}[h]
	\centering
	\begin{tikzpicture}
		\begin{class}{App}{3, 5}
			\operation{showVolunteers()}
			\operation{showShifts()}
			\operation{showPortal()}
		\end{class}

		\begin{class}{Shifts}{0, 1}
			\operation{listAvailableShifts(user : User)}
		\end{class}

		\begin{class}{Portal}{6, 1}
			\operation{showLogin()}
			\operation{showRegistration()}
		\end{class}

		\begin{class}{Store}{0, 9}
			\attribute{user : String}

			\operation{login()}
			\operation{logout()}
			\operation{register()}
			\operation{takeShift()}
			\operation{cancelShift()}
		\end{class}

		\begin{class}{Volunteers}{6, 9}
			\operation{listVolunteers()}
			\operation{listShifts(user : User)}
		\end{class}

		\begin{class}{User}{-3.5, 12}
			\attribute{name : String}
			\attribute{password : String}
		\end{class}

		\begin{class}{Showing}{3.5, 12}
			\attribute{date : Date}
		\end{class}

		\begin{class}{Shift}{0, 16}
			\attribute{date : Date}
		\end{class}

		\begin{class}{Movie}{6, 16}
			\attribute{title : String}
			\attribute{duration : Number}
			\attribute{booth : Number}
		\end{class}

		\unidirectionalAssociation{User}{users}{0..*}{Store}
		\unidirectionalAssociation{Showing}{showings}{0..*}{Store}
		\composition{App}{shifts}{1}{Shifts}
		\composition{App}{portal}{1}{Portal}
		\unidirectionalAssociation{Store}{store}{1}{App}
		\unidirectionalAssociation{Shift}{shifts}{0..*}{Store}
		\composition{App}{volunteers}{1}{Volunteers}

		\association{Showing}{movies}{1..*}{Movie}{1..*}{showing}
		\association{User}{shift}{0..*}{Shift}{1}{user}
		\association{Showing}{shift}{1}{Shift}{1}{showing}

	\end{tikzpicture}
	\caption{UML Diagram}
\end{figure}


\section*{API}

The OpenAPI specification file can be found in the
\hyperref[openapi]{appendix}.
The specification tries to follow industry ``best practices''
such as using all of the HTTP verbs
and having endpoints be plural nouns.
This already goes a long way to reaching level 2
in the Richardson maturity scale.
Otherwise the API tries to follow
how a relational database might store the data
and tries to minimize duplicate data retrieval.
Finally, as a last rule of thumb,
paths were kept shorter rather than longer.

Some notes:
\begin{itemize}
	\item User \texttt{name}s are (supposed to be) unique.
	\item Showings and shifts are identified by a unique \texttt{ID}.
	\item The \texttt{showings} endpoint returns an array of arrays
	      because it gives movie showings first by showing/shift ID
	      and then by booth number.
	      The individual movies should really also each have a showing time attached,
	      but the rationale here was that this is actually a derived value to be calculated
	      from the order and duration of the movies along with the opening times of \textsc{LocalCinema} .
\end{itemize}

\section*{Implementation}

The MVP can be tried out at this link:
\begin{center}
	\url{https://frlis21.github.io/OnlineVagtplan/}
\end{center}
There is only the frontend, there is no backend or API.
Originally the plan was to pull data from an API
but it was decided that just having a frontend
would be easier on the examinators.
Randomized test data is generated instead.

No data is saved whatsoever,
but the author claims that the frontend was purposefully designed
to make the introduction of an API trivial
with only some minor refactoring.

\subsection*{Testing}

As was maybe hinted from the UML diagrams from before,
there are no concrete classes in our implementation of \textsc{OnlineVagtplan}.
The framework used uses a component composition paradigm
instead of an object-oriented,
so we shall say components are equivalent to classes.

Unfortunately, due to time constraints
and \textsc{JavaScript} stupidity,
all attempted testing frameworks refused to work properly.
Only trivial tests could be implemented,
so there is only one test in the whole codebase.
The code coverage for our lonely little test can be found the
\hyperref[appendix:coverage]{appendix}.

As for cyclomatic complexity,
the CLI tool \texttt{scc} was used
which outputs a complexity estimate
among other fun
(albeit useless)
bits of data.
A quick scan through the source code
reveals that \texttt{scc}'s complexity estimates
are fairly accurate.

\begin{center}
	\label{cmd:scc}
	\begin{verbatim}
-------------------------------------------------------------------------------
Language                 Files     Lines   Blanks  Comments     Code Complexity
-------------------------------------------------------------------------------
JSX                         10       601       41         2      558          5
JavaScript                   3       162       13         6      143          7
-------------------------------------------------------------------------------
Total                       13       763       54         8      701         12
-------------------------------------------------------------------------------
Estimated Cost to Develop (organic) $18,603
Estimated Schedule Effort (organic) 3.03 months
Estimated People Required (organic) 0.55
-------------------------------------------------------------------------------
Processed 22015 bytes, 0.022 megabytes (SI)
-------------------------------------------------------------------------------
\end{verbatim}
\end{center}

Regarding the coverage results,
it is unclear how one small test
reaches so much of the codebase.
According to the results,
the pages components are understandably the least tested;
only one page component is really being tested after all.
Surprisingly, the store provider component sees good coverage
despite the tested page not touching it very much.
Overall, it could have been more interesting to get the testing framework to work,
but alas.

\section*{Documenting the architecture}

\begin{figure}[h]
	\centering
	\includegraphics[width=\textwidth,height=0.9\textheight,keepaspectratio]{C41}
	\caption{C4 level 1 --- Context}
\end{figure}

\begin{figure}[h]
	\centering
	\includegraphics[width=\textwidth,height=0.9\textheight,keepaspectratio]{C42.eps}
	\caption{C4 level 2 --- Container}
\end{figure}

\begin{figure}[h]
	\centering
	\includegraphics[width=\textwidth,height=0.9\textheight,keepaspectratio]{C43}
	\caption{C4 level 3 --- Component}
\end{figure}

\begin{itemize}
	\item \textbf{Layers} ---
	      The architecture of the MVP is at comprised of at most 2 layers.
	      The reactive framework used makes it easy to take advantage of the MVC model,
	      but in our MVP we tightly coupled the model and controller for testing.
	      The ultimate goal, however, was to have 3 layers,
	      where the frontend makes API calls to a backend for the data to show.
	\item \textbf{Distribution patterns} ---
	      The MVP does not make use of Renzel's distribution patterns,
	      but the final product could seem like it would use the
	      \emph{remote user interface} pattern,
	      where all of the presentation layer is packaged with the frontend
	      which communicates with the backend over a REST API.

	\item \textbf{Agile principles} ---
	      Considering a group of 1 built the entire MVP,
	      it is hard to say whether the process
	      of designing and implementing the product
	      lived up to Agile standards.
	      Regarding the simplicity and ``technical excellence''
	      points of the Agile principles,
	      it is debatable whether the MVP lives up to them.
	      The author believes the codebase is simple enough
	      once you understand the fine-grained reactivity of the framework used,
	      but the author also definitely neglected some parts of the code.
\end{itemize}

\section*{User Interface}

A fairly nice user interface naturally followed
from choosing a reactive frontend framework
to implement \textsc{OnlineVagtplan} with.
There are some visual inconsistencies,
but this is merely because the author
does not know all of the tricks to HTML and CSS yet.

There are no screenshots in this report,
but it is easy to try out the website for yourself!
Again we provide the link to the website:
\begin{center}
	\url{https://frlis21.github.io/OnlineVagtplan/}
\end{center}
The idea behind the user interface,
as with most user interfaces,
was to create an intuitive interface with as few buttons as possible
while still retaining as much functionality as possible.
No particular design principles were followed,
though the interface may have benefited from them.
The interface is simply the product of the author's intuition
of a simple user interface,
along with some inspiration from
\href{https://sourcehut.org/}{sr.ht}'s
website.

