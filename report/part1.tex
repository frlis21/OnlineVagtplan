\chapter{Preparing our process}

First we will reiterate the information gathered at the interview with the chairman
along with new information acquired during follow-up questions
so that all of our assumptions are in place.
An annotated version of the original interview can be found in the \hyperref[appendix:interview]{appendix}.

\begin{itemize}
	\item \textsc{LocalCinema} is a medium-sized establishment
	      with 6 viewing rooms and enough showings
	      such that the ticket counter and candy stand need to be continously manned.
	      Cleaners will also mostly be continuously engaged.

	\item The cinema is open from 16:00 to 22:00
	      every weekday except Monday and Tuesday
	      and of course holidays.

	\item 1 shift is the full 6 hours the cinema is open.

	\item Showings are within stretches of 15-minute blocks.
	      There are always at least 30 minutes
	      between the end of a movie and the next showing
	      to allow for cleaning.

	\item The volunteers are currently divided into the following teams:

	      \begingroup
	      \begin{tabularx}{0.9\linewidth}{p{8em}X}
		      \bfseries\raggedleft Cleaners                      &
		      Clean viewing rooms and bathrooms after a viewing.
		      At least 5 cleaners is nice to have, more is better.
		      \\
		      \bfseries\raggedleft Salespeople                   &
		      Man the ticket counter and candy stand.
		      Between 3 and 6 salespeople are preffered.
		      \\
		      \bfseries\raggedleft Planning and public relations &
		      Plan showings, events, and decoration of cinema
		      at predetermined meetings.
		      \\
		      \bfseries\raggedleft Facilities services           &
		      General handypeople, keep the building tidy
		      and repair anything gone to disrepair.
		      Only 1 or 2 are required per shift.
		      \\
		      \bfseries\raggedleft Technicians                   &
		      Man the projectors and other necessary equipment for a showing.
		      Only 1 is required per shift.
		      \\
		      \bfseries\raggedleft IT                            &
		      Small team operating on-demand,
		      responding to operational errors in the ticketing system
		      and other \textsc{LocalCinema} IT systems.
	      \end{tabularx}
	      \endgroup

	\item Supers are team leaders, with 3--6 supers per team.
	      Supers are in charge of teams in 2-week shifts;
	      customarily the supers in each team take turns with shifts,
	      but this is not always convenient for all Supers
	      so Supers agree off-hand who takes what shifts.

	\item Supers call role to make sure volunteers actually show up for shifts.

	\item Communications take place off-platform.
	      This could for example be by e-mail, group SMS,
	      a third-party application such as \textsc{GroupMe},
	      and so on.

	\item Most volunteers are at least 35 years old,
	      that is, they are mainly stay-at-home parents with schoolchildren or retired.
	      There are, however, some some active younger volunteers from the local high school
	      serving out their volunteer hours for some club participation requirements.

	\item \textsc{LocalCinema}'s ticketing system
	      provides an API for fetching showings.
\end{itemize}

\section*{Stakeholders}

Here we provide a classification of some stakeholders of the project
according to the salience model
along with a short description of each.
We do not analyze all stakeholders for the project here
because there are many that are unimportant.

\medskip

\begingroup
\renewcommand{\arraystretch}{1.1}
\renewcommand{\multirowsetup}{\raggedleft\bfseries}
\noindent\begin{tabularx}{\linewidth}{p{8em}X}
	\toprule
	\multirow[t]{2}{=}{Supers}
	 & \scshape definitive $\cdot$ core
	\\ &
	The Supers are our primary stakeholders
	as the goal of the project is to make their jobs easier.
	\\ \midrule
	\multirow[t]{2}{=}{Volunteers}
	 & \scshape expectant $\cdot$ dependent
	\\ &
	Another goal of the project is
	to make it easier for volunteers to sign up for shifts,
	but ultimately it is the Supers that decide what goes.
	\\ \midrule
	\multirow[t]{2}{=}{Authorities}
	 & \scshape latent $\cdot$ dormant
	\\ &
	There may be some government regulations we need to follow
	as volunteers will be creating accounts
	which we need to store data for.
	\\ \midrule
	\multirow[t]{2}{=}{Volunteer organizations}
	 & \scshape latent $\cdot$ discretionary
	\\ &
	Besides the \textsc{LocalCinema} association,
	other volunteer organizations may have a stake in this project
	as many parts of the project can be reused for shift scheduling elsewhere.
	It could be beneficial to build a generic shift scheduler
	in such a way that the benefits of a highly specialized solution are retained.%
	%\footnote{To expand upon this idea,
	%	consider the to-do list app \href{https://trello.com/}{\textsc{Trello}}.
	%	Assuming there are, say, less than 50 volunteers maintaining \textsc{LocalCinema},
	%	an acceptable scheduling solution requiring minimal training could be procured
	%	using \textsc{Trello}'s premium automation features and possibly a Python script
	%	(which could feasibly be procured by \textsc{LocalCinema}'s IT team).
	%	This is a generic solution with problems
	%	that a specialized solution solves.}
	\\ \midrule
	\multirow[t]{2}{=}{Infrastructure hosts}
	 & \scshape latent $\cdot$ dormant
	\\ &
	To reduce costs,
	\textsc{OnlineVagtplan} will likely be deployed with cloud service providers
	which frequently have policies and terms that need to be followed.
	\\ \midrule
	\multirow[t]{2}{=}{Cybercriminals}
	 & \scshape expectant $\cdot$ dangerous
	\\ &
	\textsc{OnlineVagtplan} needs to be secure from cybercriminals.
	\\ \bottomrule
\end{tabularx}
\endgroup

\section*{Users}

\begin{itemize}
	\item \textbf{Volunteers} are our end users;
	      they will be using \textsc{OnlineVagtplan}
	      to view and sign up for shifts.
	      As the chairman said,
	      volunteers are mainly stay-at-home parents with children going to school or retired,
	      so most volunteers are at least 35 years old
	      with some high school students in the mix.

	\item \textbf{Supers} are super users;
	      they need extra functionality to do their jobs effectively.
	      They moderate the volunteers and keep them updated,
	      and they need to be kept up-to-speed while they are on call.

	\item The \textbf{IT team} are also super users;
	      they are the tech wizards that will be interacting directly with the system
	      to support users and mitigate cyber-attacks.

	\item The \textbf{mobile app} developed by a third-party is a system user,
	      and so is the website \textbf{frontend} developed by us;
	      they will be using \textsc{OnlineVagtplan}'s API
	      to interact with the system.

	\item \textsc{LocalCinema}'s \textbf{ticketing system}
	      could be considered a system user for providing \textsc{OnlineVagtplan}
	      with the necessary data for shift scheduling.
\end{itemize}

\section*{Backlog}

%\renewcommand{\arraystretch}{1.1}

%\newcommand{\starsno}{\faStar[regular]\faStar[regular]\faStar[regular]}
%\newcommand{\starsi}{\faStarHalf*\faStar[regular]\faStar[regular]}
%\newcommand{\starsii}{\faStar\faStar[regular]\faStar[regular]}
%\newcommand{\starsiii}{\faStar\faStarHalf*\faStar[regular]}
%\newcommand{\starsiv}{\faStar\faStar\faStar[regular]}
%\newcommand{\starsv}{\faStar\faStar\faStarHalf*}
%\newcommand{\starsvi}{\faStar\faStar\faStar}

\newlength{\cookieslen}
\newcommand{\cookiesvi}{\makebox[\width][l]{\faCookie\,\faCookie\,\faCookie}}
\settowidth{\cookieslen}{\cookiesvi}
\newcommand{\cookiesi}{\makebox[\cookieslen][l]{\faCookieBite}}
\newcommand{\cookiesii}{\makebox[\cookieslen][l]{\faCookie}}
\newcommand{\cookiesiii}{\makebox[\cookieslen][l]{\faCookie\,\faCookieBite}}
\newcommand{\cookiesiv}{\makebox[\cookieslen][l]{\faCookie\,\faCookie}}
\newcommand{\cookiesv}{\makebox[\cookieslen][l]{\faCookie\,\faCookie\,\faCookieBite}}

Requirements analysis essentially consisted of the author asking himself
``If I were an $X$, what would I want out of \textsc{OnlineVagtplan}?''
where $X$ is a user group for all user groups outlined above.
From this, a long list of features in the form of user stories was created for each user group.

In celebration of the holidays,
the author has assigned to each of these features
some amount of cookies you get upon completion of the feature
from \faCookieBite\ (half a cookie) to \cookiesvi\ (three full cookies)
where you get more cookies the more important a feature is.
Then you get the most cookies by completing system-critical features.

A weight is given to each feature
from a subsequence of the almost-fibonacci numbers \[(1, 2, 3, 5, 10, 20).\]
This weight is calculated purely from intuition
and is a rough estimate of the difficulty
and amount of work needed to implement the feature.

Finally, each feature is assigned a short name
and each user story is categorized into epics
for easier referencing.
Note that some epics and feature names are associated with several user groups.

\begin{xltabular}{\linewidth}{lcrX}
	\toprule
	\scshape name & \scshape cookies & \faWeightHanging & \scshape story \\
	\midrule
	\endhead
	\bottomrule
	\endlastfoot

	\multicolumn{3}{l}{\itshape\bfseries Account\ldots} & \itshape\bfseries As a volunteer, I want\ldots \\*

	Signup & \cookiesvi & 2 & to create an account
	so that I can volunteer. \\*

	Login & \cookiesvi & 5 & to log in
	so that I can do stuff. \\*

	OAuth & \cookiesi & 10 & to log in with \textsc{Google} or \textsc{Facebook}
	so that I don't have to create yet another account for something. \\

	\midrule[0pt]
	\multicolumn{3}{l}{\itshape\bfseries Shifts\ldots} \\*

	Filter & \cookiesiv & 20 & to search and select shifts by time
	so that I can find shifts that fit my schedule. \\*

	Take & \cookiesvi & 2 & to sign up for shifts
	so that \textsc{LocalCinema} can stay open. \\*

	Select & \cookiesiii & 2 & to sign up for shifts in bulk
	so that I can get it over with quickly. \\*

	Receipt & \cookiesiv & 5 & to receive some kind of confirmation when I sign up for a shift
	so that I can be sure I have signed up for that shift. \\*

	Status & \cookiesvi & 1 & to see my shifts
	so that I can remember them. \\*

	Cancel & \cookiesvi & 1 & to cancel a shift
	so that I can tend to unexpected events. \\

	\midrule[0pt]
	\multicolumn{2}{l}{\itshape\bfseries Notify\ldots} \\*

	Help & \cookiesv & 10 & to be notified when my teams' upcoming shifts are unfilled
	so that I can potentially fill in in time. \\*

	Cancellation & \cookiesiv & 2 & to be notified when a shift is canceled
	so that I don't waste time. \\*

	Event & \cookiesi & 3 & to be notified of events
	so that I can join them. \\*

	Reminder & \cookiesv & 10 & to be notified when I have an upcoming shift
	so that I don't forget it. \\*

	Settings & \cookiesii & 2 & to edit my notification settings
	so that I am not spammed. \\

	\midrule[0pt]
	\multicolumn{2}{l}{\itshape\bfseries Team\ldots} \\*

	Join & \cookiesvi & 2 & to join a team
	so that I can do volunteer work I am interested in. \\*

	Super & \cookiesv & 1 & to see the current Super on duty
	so that I can contact them. \\*

	Members & \cookiesi & 1 & to see my teammates
	so that I can contact them. \\*

	Signups & \cookiesiv & 1 & to see who is signed up for a shift
	so that I can sign up with my buddies
	and harbor grudges against those that cancel right before a showing. \\

	\midrule[0pt]
	\multicolumn{2}{l}{\itshape\bfseries Profile\ldots} \\*

	Picture & \cookiesi & 2 & to upload a profile picture
	so that I can be stylish. \\*

	Stats & \cookiesv & 1 & to see my current rewards and statistics
	so that I can marvel at my philanthropy. \\*

	Rewards & \cookiesv & 5 & to cash in rewards
	so that I can enjoy them. \\

	\midrule[0.01pt]
	\multicolumn{3}{l}{\itshape\bfseries Announcements\ldots} & \itshape\bfseries As a Super, I want\ldots \\*

	News & \cookiesi & 10 & to pin information on a ``bulletin board''
	so that users can stay updated. \\

	\midrule[0pt]
	\multicolumn{3}{l}{\itshape\bfseries Notify\ldots} \\*

	Cancellation & \cookiesv & 20 & to be notified when a shift is canceled right before a showing
	so that I can fill in. \\

	\midrule[0pt]
	\multicolumn{3}{l}{\itshape\bfseries Moderation\ldots} \\*

	Ban & \cookiesiv & 5 & to ban a user
	so that they don't cause trouble. \\*

	Promote & \cookiesv & 5 & to promote users
	so that there can be more Supers. \\*

	Profiles & \cookiesi & 2 & users to have a profile picture
	so that I can easily distinguish between them. \\*

	Contact & \cookiesii & 5 & users to have an e-mail address or phone number
	so that I can contact them. \\

	\midrule[0.01pt]
	\multicolumn{3}{l}{\itshape\bfseries Support\ldots} & \itshape\bfseries As an IT administrator, I want\ldots \\*

	Reset & \cookiesv & 2 & to reset a user's email or password so that they can log in again. \\*

	Edit & \cookiesvi & 3 & to edit users' stats and rewards
	so that data can be migrated. \\*

	Logs & \cookiesiii & 5 & to read logs so that I can potentially fix operational errors. \\*

	Backups & \cookiesv & 10 & to backup and restore data
	so that \textsc{LocalCinema}'s operations can quickly recover. \\

	\midrule[0pt]
	\multicolumn{3}{l}{\itshape\bfseries Mitigation\ldots} \\*

	Emergency & \cookiesii & 10 & to turn off some features
	so that they cannot be abused. \\

	\midrule[0.01pt]
	\multicolumn{3}{l}{\itshape\bfseries Stability\ldots} \\*

	Uptime & \cookiesv & 5 & Have $99.5\%$ uptime
	so \textsc{LocalCinema} can stay operational. \\*

	\midrule[0pt]
	\multicolumn{3}{l}{\itshape\bfseries Experience\ldots} \\*

	Simple & \cookiesv & 10 & Have a simple UI/UX
	for the boomers.
\end{xltabular}


This would ideally be the backlog for the complete product,
but there are more than likely some items missing
which will be discovered during the course of development.
This backlog has a combined weight of 180,
but we will only be creating the MVP for this sprint.
For this, these selected features from selected epics
will be implemented:
\begin{center}
	\begin{tabular}{lll}
		\textbf{Account} & \textbf{Shifts} & \textbf{Teams} \\
		\midrule
		Signup           & Take            & Join           \\
		Login            & Cancel          &                \\
		                 & Status
	\end{tabular}
\end{center}
for a combined weight of 13.
That is, the MVP is $13/180^{\mathrm{ths}} \approx 7\%$ of our product
according to our full backlog.

\subsection*{But why}

The backlog is structured as it is
because (user) stories are a super nice and simple way
of communicating system requirements.
Any fool can infer ``sub-requirements'' for features described by a story,
especially if said fool is the only one working on the project
(as in this case).

There are six priority levels
mainly because three is too little,
but it is also a nice number of cookies to have,
so each priority level is half of a cookie.
Six levels is also a nice balance
between coarse and granular prioritization.

Features are assigned a weight
mainly for fun so the author can feel like he is getting work done
when he completes a feature.
They can also be used to calculate approximately how ``done''
the product is, as was done above.
