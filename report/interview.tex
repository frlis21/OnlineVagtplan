\chapter{Annotated interview}
\label{appendix:interview}

\begingroup

\newcommand{\dialogue}[3]{%
	\scshape\footnotesize #1 \rule{0pt}{3ex} \\*%
	\setlength{\parskip}{4pt}%
	#2 & \itshape #3}

\noindent\begin{longtable}[l]{p{0.7\linewidth}p{0.3\linewidth}}
	\toprule
	\endfirsthead
	\midrule
	\endhead
	\midrule
	\endfoot
	\bottomrule
	\endlastfoot

	\dialogue{Consultant}{%
		Hello \textsc{Chairman}.
		I'm glad you had some time to talk to me today.
		Can you share some of the background and history of \textsc{LocalCinema}?
	}{}
	\\
	\dialogue{Chairman}{%
		Of course.
		It is my pleasure.
		In the last 30 years, this cinema has existed in this city.
		But, due to some unfortunate financial events 5 years ago the former owner had to step back and close the cinema.
		The community in the city felt it was important to get the cinema to open again.
		This was why they decided to start this association with the sole purpose of operating the local cinema voluntarily without any paid employees.

		The association has now been operating the cinema since the founding and we still experience a rise in the number of volunteers and paying visitors.
		We are however also seeing the same volunteers taking shifts, again and again, making them spend quite a lot of their spare time on this work.
		We would like to get more members to volunteer for a shift
		and we hope by making this accessible to the members online that it will lower the bar sufficiently to achieve this.
	}{}
	\\
	\dialogue{Consultant}{%
		Who signs up as members of \textsc{LocalCinema}?
	}{}
	\\
	\dialogue{Chairman}{%
		A quite diverse group to be honest.
		\textbf{Generally speaking, the average member is more than 35 years old.}
		\textbf{Unfortunately, not all members own a computer at home.}

		Some of our members function as facility service making sure the building is not in decay.
		The background of these is often as carpenters, structural engineers or similar.
		We also have salespeople manning the ticket stand and candy stand.
		They don't share a common background other than being quite talkative and extrovert.
		We also have some technical staff taking care of operating the projector and sound equipment.
		The cleaners are also volunteers.
		I have tried to convince them to clean my house as well, but so far no luck.
		The PR group has the responsibility to create decorations for our windows and function as event planners for larger events as well.
	}{%
		This group is not very tech-savvy, so a simple UX is a priority.
		They probably at least own a phone so an app would be most convenient.
	}
	\\
	\dialogue{Consultant}{%
		Can anybody become a member?
	}{}
	\\
	\dialogue{Chairman}{%
		Yes as long as you are willing to help voluntarily.
		We prefer if our members are older than 18 years.
		\textbf{All members are in one or more groups.}
		Nothing is stopping you from being both in the group managing the tickets and the cleaning group.
		\textbf{A side-effect of this is that the groups are tightly knitted together.}
	}{So communication happens by a third party such as group SMS.}
	\\
	\dialogue{Consultant}{%
		Can you tell me how you enlist for a shift currently?
	}{}
	\\
	\dialogue{Chairman}{%
		We currently have lists hanging in the cinema where people then write their names on the individual shows.
		The problem is people sometimes forget about their shifts and then we might end up without anybody to start the movie or sell tickets.
		This doesn't happen that often, but it has happened previously.
		Some members also state it is a blocker for them that they need to go to the cinema to enlist for a shift, resulting in they never take a shift.
		I truly believe we could get a lot more volunteers with more buy-in from the members if they could easily enlist for shifts.
	}{}
	\\
	\dialogue{Consultant}{%
		What do you do if not all shifts are taken by members?
	}{}
	\\

	\dialogue{Chairman}{%
		\textbf{If that happens we have the aptly called Supers.}
		They make sure all shifts are filled.
		What they often do is to look forward in time and call around if a shift is still empty.
		Some of these Supers have been spending a lot of time on this.
		Another problem is they physically need to go to the cinema to check the lists.
		If they can't find a volunteer to that shift, then they need to man it themselves
		which is of course not sustainable.

		By the way, people get tickets for the shows every half year depending on the number of shifts they have had.
		This is a small incentive to motivate people.
		They get one ticket per 5 shifts they have had.
		It is the job of the Supers to distribute these tickets and keep a ledger on how many shifts each member have had.
		\textbf{It would be nice if the system could keep track of this on its own.}
		Again, this would free up a lot of time for the Supers.
	}{Supers $=$ Team leaders.}
	\\
	\dialogue{Consultant}{%
		Let's say a volunteer gets sick, how do they notify the Supers? What if you just want to cancel a shift?
	}{}
	\\
	\dialogue{Chairman}{%
		You can only do this by contacting a Super directly.
		\textbf{This can be cumbersome because people need to figure out who is the Super on call at the moment.}
		\textbf{We don't like to see people canceling their shifts less than a week before the show,}
		simply because we can't find a replacement within such a short time.
	}{%
		When a shift is canceled, the Supers and volunteers need to be notified.
	}
	\\
	\dialogue{Consultant}{%
		Who are the Supers and how do you become one?
	}{}
	\\
	\dialogue{Chairman}{%
		The Supers are members of the groups that are up for an extra challenge.
		Each group has 3 to 6 Supers.
		A Super is on call for 2 weeks at the time.
		\textbf{You need to be in the group for at least one year and have had more than 30 shifts before you can come into consideration as a Super.}
		{So per-volunteer stats need to be tracked.}
	}{}
	\\
	\dialogue{Consultant}{%
		How do you know which movies are currently playing in the cinema?
	}{}
	\\
	\dialogue{Chairman}{%
		We have an electronic ticketing system with all the shows listed.
		Our website pulls the data from that system directly.
		\textbf{It would be nice if the OnlineVagtplan can import the data automatically as well.}
		This would avoid a lot of double bookkeeping.
	}{}
	\\
	\dialogue{Consultant}{%
		I think I'm starting to have an idea of what we need to focus on.
		I'll take these inputs home and try to come up with a solution.
		I'll create a writeup and send it to you if you want.
		If you think of anything we forgot to talk about just reach out me and we can get it documented.
		I might also have some clarifying questions I would like to ask you at some point.
	}{}
\end{longtable}

\endgroup
